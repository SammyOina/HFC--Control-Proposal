\section{Introduction}
\label{sec:introduction}
\subsection{Background}
(Insert your content)

gghjbbnmmm

\subsection{Problem statement}
\paragraph{}As a measure to curb pollution due to the industrialization and transportation sectors, world governments are turning to alternative sources of energy. These alternative sources of energy should drastically reduce the pollution rates by cutting down emissions. Fuel cell technology is one such example of alternative sources of energy. A fuel cell uses the chemical energy of hydrogen or other fuels to cleanly and efficiently produce electricity. Moreover, fuel cells can operate at higher efficiencies than combustion engines and can convert the chemical energy in the fuel directly to electrical energy with efficiencies capable of exceeding 60\%. Fuel cells have lower or zero emissions compared to combustion engines.
\paragraph{}The various departments of energy, however,  have to work closely with national laboratories, universities, and industry partners to overcome critical technical barriers to fuel cell development. These barriers are cost, performance, and durability which are still key challenges in the fuel cell industry. 
\paragraph{}This design proposal seeks to provide a solution to improving the fuel cell’s performance by improving the robustness and efficiency of the Fuel Cell stack system for real world conditions through precise control of reactant flow and pressure, stack temperature, and membrane humidity.
\subsection{Objectives}
(Insert your content)
\subsection{Justification of the study}
\paragraph{}Additive manufacturing offers the ability to produce intricate products and parts with lower development costs, shorter lead times, less energy consumed during manufacturing as well as less material waste. This method can be used to manufacture delicate components such as the bipolar plates with elimination of the risks involved such as breakage of brittle Graphene material during production.     
\paragraph{}Precise control of reactant flow and pressure, stack temperature, and membrane humidity will increase the fuel cell’s robustness as well as efficiency.
\paragraph{}The goal of this research is to develop physic-based dynamic models of fuel cell systems and fuel processor systems and then apply multivariable control techniques to study their behavior. The analysis will give insight into the control design limitations and provide guidelines for the necessary controller structure and system re-design.